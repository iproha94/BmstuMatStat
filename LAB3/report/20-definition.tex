\chapter{Определения}

\section{Постановка задачи аппрокисмации неизвестной зависимости по результатам наблюдений}
Переменная Y стохастически зависит от переменных $X_1,...,X_p$ если на изменение этих переменных Y реагирует изменением своего закона распределения. Задача, связанная с изучением стохастических зависимостей между слуйчайной величиной Y и детерминированными величинами $X_1,...,X_p$ носящих колличественный характер, составляют премет исследования регрессионного анализа. В регрессионном анализе используют модель черного ящика, как наиболее общую модель, связанную с понятием отображение $\Psi$.
На вход поступает вектор $(X_1,...,X_p)$ , который посредством некоторого отображения $\Psi$ и случайных возмущений $(\varepsilon_1,...,\varepsilon_m)$ преобразуется в вектор $(Y_1,...,Y_m)$ .

\section{Понятие МНК-оценки параметров линейной модели}

Для простоты ограничемся случаем $p = m = 1$. Предположим, что в нашем распоряжении имеется n результатов наблюдений.
\begin{equation}
 \begin{cases}
   y_1 = \Phi(t_1) + \varepsilon_1\\
   ...
   y_n = \Phi(t_n) + \varepsilon_n\\
 \end{cases}
\end{equation}
Требуется на основании этих данных подобрать функцию $\hat{\Phi}(t)$ таким образом, чтобы она наилучшим образом аппроксимировала (описывала) функцию $\Phi(t)$. Часто в качестве функции $\Phi(t)$ рассматривают линейную комбинацию некоторых функций $\psi_1(t),...,\psi_s(t)$:
$$\Phi(t) = \Theta_1\psi_1(t) + ... + \Theta_s\psi_s(t)$$

Оценка $\vec{\hat{\Theta}}$ вектора $\vec{\Theta}$ называется МНК оценкой если она доставляет минимальное значение функционалу
$$S(\vec{\Theta}) = ??? $$


