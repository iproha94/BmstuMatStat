\chapter{Определения}
\section{Выборочная функция распределения}
\begin{equation}
\hat{F}(x, \vec{X}) = \frac{n(x, \vec{X})}{n}
\end{equation}

где 

$n(x, \vec{X})$ - функция, которая для каждой реализации $\vec{x}$ случайной выборки $\vec{X}$ принимает значение, равное $n(x, \vec{x})$,

$n(x, \vec{x})$ - число элементов выборки $\vec{x}$ значения которых меньше чем $x$

\section{Выборочная плотность распределения}

\begin{equation}
f_x(x) = 
 \begin{cases}
   \frac{n_i}{n \cdot \Delta}, &\text{$x \in J_i$}\\
   0, &\text{$x \notin [x_{(1)}, x_{(n)}]$}
 \end{cases}
\end{equation}

где 

$n_i$ - число элементов выборки $\vec{x}$ попавших в $J_i$

$\Delta = \frac{x_{(n)} - x_{(1)}}{m}$

$m = \log_{2} n + 1$

\section{Гистограмма}
График функции $f_x(x)$ называется гистограммой выборки $\vec{x}$.
